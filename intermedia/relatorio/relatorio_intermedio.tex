\documentclass[a4paper]{article}

\usepackage[portuguese]{babel}
\usepackage{comment}
\usepackage[T1]{fontenc}
\usepackage[utf8]{inputenc}
\usepackage{hyperref}
\usepackage{graphicx}
\usepackage{float}
\usepackage{multirow}
\usepackage{indentfirst}
\usepackage[hypcap]{caption} % makes \ref point to top of figures and tables

\begin{document}

\begin{titlepage}

	\begin{center}

		\includegraphics[width=6cm]{./title}\\[3cm]

		\textsc{\LARGE Redes Móveis e Sem Fios}\\[1.5cm]

		\textsc{\Large Relatório Final  }\\[1.5cm]
		
		
		\textsc{Development of Internet of Things sensor monitoring based on SigFox, Arduino and Android }\\[1.5cm]
		



		


		\noindent
		\begin{minipage}{0.4\textwidth}
			\begin{flushleft} \large
				Bernardo Gomes, 75573
			\end{flushleft}
		\end{minipage}
		\begin{minipage}{0.4\textwidth}
			\begin{flushright} \large
				Diogo Martins, 75462
			\end{flushright}
		\end{minipage}

		\vfill

		{\large \today}


	\end{center}

\end{titlepage}
\hypersetup{%
    pdfborder = {0 0 0}
}
\pagenumbering{arabic}

\section{Objectivo}

O objectivo do projecto é o desenvolvimento de um sistema de monitorização de temperatura. 

O sistema, deverá ser baseado num sensor de temperatura associado a um dispositivo \textit{arduino (akeru 3.3)}, que irá comunicar as suas medições a um servidor \textit{Sigfox}, armazenando-as na \textit{cloud}.

Na óptica do utilizador, irá ser desenvolvida uma aplicação em ambiente \textit{android}, que fornecerá os dados presentes na \textit{cloud} com uma apresentação \textit{user friendly}. Pretende-se ainda que seja possível que o utilizador registe um novo dispositivo a monitorizar na aplicação, bem como definir alarmes para certos valores de temperatura.

\section{Arquitectura do projecto}
\section{Trabalho realizado}
\section{Considerações finais}

\end{document}